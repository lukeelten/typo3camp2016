% !TeX spellcheck = en_US
\section{Composer - Dependency Management for PHP}

\subsection{Why to use a dependency manager?}
- never forget to install a dependency
- start using a project on a fresh platform automatically
- keep everything up-to-date
- keep track of what you need and what your dependencies need


\subsection{Examples in other languages}
- maven (Java)
- gem / bundler (Ruby)
- cocoapods (Apple)
- pip (Python)
- npm / bower (Javascript, CSS)

\subsection{How to use composer}

For using composer, a locally installed php version (>= 5.3.2) required.
The installation on Linux and Macs is quite easy. The following command downloads the composer PHP Archive (composer.phar) and stores it in the current directory. 
\begin{lstlisting}[language=bash]
$ curl -sS https://getcomposer.org/installer | php
$ ./composer.phar
$ sudo mv composer.phar /usr/bin/composer
$ composer
\end{lstlisting}
\captionof{lstlisting}{Bash commands to download and install composer}

The installation on windows is also easy when there is an existing php installation with a working openssl extension. How to install php is out of the scope of this talk, please visit \url{www.php.net}.
The composer file can easily be downloaded at \url{getcomposer.org} and can be used out of the working directory or can be placed inside PATH. 



- show start script for new composer project
- show installation of some famous dependency
