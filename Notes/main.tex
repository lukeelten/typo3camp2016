\documentclass[a4paper,oneside,titlepage]{article}

\usepackage[english]{babel}
\usepackage[T1]{fontenc}
\usepackage[utf8]{inputenc}

\usepackage[pdftex]{graphicx} %%Grafiken in pdfLaTeX
\usepackage{a4wide} %%Kleinere Seitenränder = mehr Text pro Zeile.
\usepackage{longtable} %%Für Tabellen, die eine Seite überschreiten
\usepackage{pdflscape}
\usepackage{caption}
\usepackage{float}
%\usepackage[tablegrid,tocentry]{vhistory}
\usepackage[tablegrid]{vhistory}
\usepackage[nottoc]{tocbibind}
\usepackage[toc,page]{appendix}
\usepackage{pdfpages}
\usepackage[rightcaption]{sidecap}
\usepackage{cite}
\usepackage[]{acronym}
\usepackage[pdftex,scale={.8,.8}]{geometry}
\usepackage{layout}
\usepackage{subfigure}
\usepackage{fancybox}
\usepackage{fancyhdr}
\usepackage[toc]{glossaries}
\usepackage[left,pagewise,modulo]{lineno}
\usepackage[pdftex,colorlinks=false,hidelinks,pdfstartview=FitV]{hyperref}
\usepackage{listings}

\definecolor{mygreen}{rgb}{0,0.6,0}
\definecolor{mygray}{rgb}{0.5,0.5,0.5}
\definecolor{mymauve}{rgb}{0.58,0,0.82}

\lstset{ %
	backgroundcolor=\color{white},   % choose the background color
	basicstyle=\footnotesize,        % size of fonts used for the code
	breaklines=true,                 % automatic line breaking only at whitespace
	captionpos=b,                    % sets the caption-position to bottom
	commentstyle=\color{mygreen},    % comment style
	escapeinside={\%*}{*)},          % if you want to add LaTeX within your code
	keywordstyle=\color{blue},       % keyword style
	stringstyle=\color{mymauve},     % string literal style
	language=PHP,
	frame=single,
	numbers=left, 
	numberstyle=\small
}

\usepackage{metainfo}

\def\Company{}
\def\Institute{\textit{Fontys School of Technology \& Logistics}}

\def\BoldTitle{Composer \& Frameworks}
\def\Subtitle{How they make development easier}
\def\Authors{Tobias Derksen}


\title{\textbf{\BoldTitle}\\\Subtitle}
\author{\Authors \\ \Institute \\ \\ \\ \\}
\date{Venlo, 22nd April 2016}



% Generiert Deckblatt automatisch
\AtBeginDocument{
	\maketitle
	\thispagestyle{empty}
}

% Generiert PDF Informationen
\hypersetup{pdfinfo={
		Title={\BoldTitle},
		Author={\Authors},
		Subject={\Subtitle}
}}

%%%%%%%%%%%%%%%%%%%%%%%%%%%%%%%%%%%%%%%%%%%%%%%%%%%%%%%%%%%%%
%% DOKUMENT
%%%%%%%%%%%%%%%%%%%%%%%%%%%%%%%%%%%%%%%%%%%%%%%%%%%%%%%%%%%%%
\begin{document}
\setcounter{secnumdepth}{0}

\DeclareGraphicsExtensions{.pdf,.jpg,.png}

\newpage


% Den Inhalt des Dokuments einbinden
% !TeX spellcheck = en_US
\section{Introduction}



% !TeX spellcheck = en_US
\section{Composer - Dependency Management for PHP}

\begin{frame}{What is Dependency Management?}
	\begin{itemize}
		\item Centralized dependency management
		\item Manage dependencies versions
		\item Automate parts of the build process
	\end{itemize}
\end{frame}

\begin{frame}{Dependency Management in other languages}
	\begin{itemize}
		\item Maven (Java)
		\item gem / bundler (Ruby)
		\item pip (Python)
		\item npm / bower (Javascript)
	\end{itemize}
\end{frame}

\begin{frame}{Why composer makes your life easier}
	\begin{itemize}
		\item Manage dependencies versions
		\item Allows automated build process
	\end{itemize}
\end{frame}


\begin{frame}{Important files}
	\begin{itemize}
		\item composer.json
		\item composer.lock
		\item vendor/autoload.php
	\end{itemize}
\end{frame}

\begin{frame}{composer.json}
	\{ ~\\
		\hspace{1cm}"name": "vendorName/projectName", ~\\
		\hspace{1cm}"require": \{ ~\\
		\hspace{2cm}"important/package": "2.*", ~\\
		\hspace{1cm}\} ~\\
	\} ~\\
\end{frame}

\begin{frame}{Versioning Schemas}
  \begin{table}
  	\begin{tabular}{|c|c|}
  		\toprule
  		Constraint & Meaning\\
  		\midrule
  		2.0.1 & exactly 2.0.1 \\
  		2.0.* & \textgreater= 2.0.0 \textless2.1.0\\
  		\textgreater= 2.0 &  \\
  		\~{}2.0 & \textgreater= 2.0.0 \textless3.0.0 \\
  		\~{}2.0.0 & \textgreater= 2.0.0 \textless2.1.0 \\
  		\^{}2.0.0 & \textgreater= 2.0.0 \textless3.0.0 \\
  		1.0 - 2.0 & \textgreater= 1.0.0 \textless2.1 \\
  		\bottomrule
  	\end{tabular}
  \end{table}
\end{frame}

\begin{frame}{Common mistakes}
	\begin{itemize}
		\item Version: \textgreater= 2.0.*
		
	\end{itemize}
\end{frame}
\section{Frameworks - Do not reinvent the wheel!}
\begin{frame}{What is a Framework?}
\end{frame}

\begin{frame}{Why should I use a Framework?}
\end{frame}

\begin{frame}{Some Frameworks in detail}
\end{frame}
% !TeX spellcheck = en_US

\section{Zend Framework}
% !TeX spellcheck = en_US

\chapter{CakePHP}
% !TeX spellcheck = en_US
\section{Symfony}


% !TeX spellcheck = en_US
\section{Flow}



\end{document}
